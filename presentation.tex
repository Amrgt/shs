\documentclass{beamer}
\usepackage{polyglossia}
\usepackage{tikz}
\usetikzlibrary{arrows,shapes}
\tikzstyle{every picture}+=[remember picture]
\setdefaultlanguage{french}

\newcommand{\theauthor}[3][3cm]{%
  \vbox{\hbox{\includegraphics[width=#1]{#2}}\hbox to #1{\hss#3\hss}}}

\usetheme{Nolo}
\author[J.Amiguet, J.Droz]{\theauthor[0.125\textwidth]{portrait.jpg}{Jérôme Amiguet}\hskip 0.2\textwidth \theauthor[0.125\textwidth]{johan.jpg}{Johan Droz}}
\title{Sticky Policies}
\date{26 november 2014}

\makeatletter
    \newenvironment{withoutheadline}{
        \setbeamertemplate{headline}[default]
        \setbeamertemplate{footline}[default]
        \def\beamer@entrycode{\vspace*{-\headheight}}
    }{}
\makeatother

%\usepackage[francais]{babel}

\DeclareTextFontCommand{\emph}{\bf\color{navyblue}}

\begin{document}
\institute[EPFL]{École Polytechnique Fédérale de Lausanne}
\titlegraphic{\includegraphics[height=25pt]{EPFL_LOG_QUADRI_Red}}
%\setbeamertemplate{footline}{}
%\setbeamertemplate{headline}{}
\begin{withoutheadline}
\begin{frame}
\titlepage
\end{frame}
\end{withoutheadline}
\setcounter{framenumber}{0}
\begin{frame}
\tableofcontents
\end{frame}
\section{Purpose}
\input{purpose}
\section{How it works}
\input{mechanisms}
\section{Actors}
\input{actors}
\section{Illustration}
\begin{frame}
\frametitle{Illustration}
\only<1>{The data is\tikz[baseline]{\node[anchor=base] (enc) {\emph{encrypted}.}}} \only<2>{The policy is added to the data and the key and its hash are\tikz[baseline]{\node[anchor=base] (ch) {\emph{encrypted}}}with the public key of the trusted authority.}\only<3>{The policy is\tikz[baseline]{\node[anchor=base] (sig) {\emph{signed}}}by the user.}
\only<4>{The service provider is\tikz[baseline]{\node[anchor=base](challenge){\emph{challenged}}}by the trusted authority to access the data.}
\only<5>{Finally, the service provider\tikz[baseline]{\node[anchor=base] (retr) {\emph{retrieves}}}the symmetric key to access the data.}
\begin{figure}
\centering
\begin{tikzpicture}
\node[anchor=south west, inner sep=0] (image) at (0,0) {\includegraphics[width=0.9\textwidth]{Mechanisms.png}};
\begin{scope}[x={(image.south east)},y={(image.north west)}]
\node (authenticate) at (0.525, 0.575) {};
\node (check) at (0.35, 0.575) {};
\node (delivery) at (0.82, 0.475) {};
\node (symkey) at (0.2, 0.4) {};
\node (encryption) at (0.26, 0.45) {};
\node (blargh) at (0.6, 0.425) {};
%\draw[help lines,xstep=.1,ystep=.1] (0,0) grid (1,1);
%\foreach \x in {0,1,...,9} { \node [anchor=north] at (\x/10,0) {0.\x}; }
%\foreach \y in {0,1,...,9} { \node [anchor=east] at (0,\y/10) {0.\y}; }
\end{scope}
\end{tikzpicture}
\caption{Core mechanisms underpinning the management of sticky policies, from \itshape{Sticky Policies: An Approach for Managing Privacy across Multiple Parties (2011)}}
\end{figure}
\begin{tikzpicture}[overlay]
\path<1>[->, thick, color=navyblue] (enc.south) edge[out=-90, in=90] (encryption);
\path<2>[->, thick, color=navyblue] (ch.south) edge[out=-90, in=90] (check);
\path<3>[->, thick, color=navyblue] (sig.south) edge[out=-90, in=90] (authenticate);
\path<4>[->, thick, color=navyblue] (challenge.south) edge[out=-90, in=90] (blargh);
\path<5>[->, thick, color=navyblue] (retr.south) edge[out=-90, in=90] (delivery);
\end{tikzpicture}
\end{frame}
\section{Usefulness for the course}
\input{usefulness}
\begin{frame}
\frametitle{End}
\centering\Huge Questions ?
\end{frame}
\end{document}